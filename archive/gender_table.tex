\begin{singlespace}
\begingroup
\renewcommand*{\arraystretch}{1.241}
\footnotesize{
\begin{center}
\begin{threeparttable}
\caption{\textrm{\normalfont \small Mean YLILI dimension and indicator scores by gender \label{tab:gendermeans}}}
\begin{tabular}{l c c c}
  \hline \hline
 & Male & Female & $\Delta$ \\ 
  \hline
\textbf{YLILI score} & 67.48 & 64.58 & 2.91 \\ 
 Transition & 82.73 & 75.56 & 7.17 \\ 
  \hspace*{1cm} Share of youth NEET & 82.32 & 66.54 & 15.79 \\ 
  \hspace*{1cm} Youth skills mismatch rate & 81.02 & 75.52 & 5.50 \\ 
  \hspace*{1cm} Relative working conditions ratio & 84.85 & 84.63 & 0.22 \\ 
  Working conditions & 62.80 & 63.55 & -0.74 \\ 
  \hspace*{1cm} Youth working poverty rate & 74.88 & 74.77 & 0.12 \\ 
  \hspace*{1cm} Youth TR underemployment rate & 90.66 & 88.90 & 1.77 \\ 
  \hspace*{1cm} Share of youth in informal employment & 11.31 & 11.60 & -0.29 \\ 
  \hspace*{1cm} Share of youth in elementary occup. & 74.36 & 78.92 & -4.56 \\ 
  Education & 56.92 & 54.62 & 2.29 \\ 
  \hspace*{1cm} Share of youth with no secondary educ. & 62.12 & 59.23 & 2.88 \\ 
  \hspace*{1cm} Youth illiteracy rate & 85.60 & 80.62 & 4.98 \\ 
  \hspace*{1cm} Harmonized test scores & 23.04 & 24.02 & -0.98 \\ 
   \hline \hline
\end{tabular}
 \begin{tablenotes}
      \small
      \item \textit{Notes}: Most recent observations, dating back no further than 2010. Rescaled indicator scores shown---higher values always correspond to better labor market outcomes. 
    \end{tablenotes}
\end{threeparttable}
\end{center}
}
\endgroup
\end{singlespace}